\documentclass[a4paper, 11pt]{article}
\usepackage[left=1.5cm,text={18cm, 25cm},top=2.5cm]{geometry}
\usepackage[utf8x]{inputenc} 
\usepackage[czech]{babel}
\usepackage[IL2]{fontenc}
\providecommand{\uv}[1]{\quotedblbase #1\textquotedblleft}

% Thesis variables
\newcommand{\thesisname}{Implementace informační bezpečnosti do ŠVP pro Informatiku a výpočetní techniku na gymnáziích}

\usepackage{amsthm}
\usepackage{amsmath}
\usepackage{amsfonts}
\usepackage{fancyvrb}
\usepackage{hyperref}
\hypersetup{
    colorlinks,
    citecolor=black,
    filecolor=black,
    linkcolor=black,
    urlcolor=black
}

\title{\thesisname}
\author{Bc. Daniel Dušek}

\begin{document}

\thispagestyle{empty}
\begin{center}
\Huge
\textsc{Fakulta informačních technologií, Vysoké učení technické v~Brně}\\
\LARGE
\vspace{\stretch{0.382}}
% Here should come title text, or what
\thesisname
\vspace{\stretch{0.618}}
\end{center}
{\Large \today \hfill Bc. Daniel Dušek}


% WORK TODO LIST
% TODO: FILL ALL THE SECTIONS MARKED WITH TODO COMMENTARIES
% ADD CORESPONDING IMAGES



\newpage
\tableofcontents

% SECTION INTRODUCTION
% PAGE COUNTER RESET
% THIS IS THEORETICAL PART
\newpage
\setcounter{page}{1}
\section{Úvod}
% TODO FILL THIS SECTION WITH SOMETHING USEFULL
% Punch-line
% Povídání o rozvoji výpočetní techniky
% -> návaznost: zavedení výuky výpočetní techniky do škol - rok 2004? Nebo 2006?
% Nutná potřeba, jako kdysi byla potřeba zavést ICT do škol, je nyní zavést diskusi o bezpečnosti technologií do škol
% Poskytovaný prostor RVP

% DEPRECATED-TEXT
%Tak jako je 19. století nazývané stoletím páry, či 20. století stoletím techniky, bude jednou 21. století pravděpodobně nazývané stoletím počítačů a sítí. S neustále se zrychlujícím rozvojem oblasti informačních technologií a zvyšující se dostupností elektroniky schopné připojovat se do Internetu, se stejně rychle objevuje a rozvíjí další fenomén.  

% Paper contents
% * povídání o rozvoji výpočetní techniky
% * návaznost - zavedení výuky výpočetní techniky do škol
% * nutná potřeba, jako kdysi byla potřeba zavést výuku ICT do škol, je nyní potřeba zavést výuku o bezpečnosti
% * poskytovaný prostor v RVP
% 
% Topic: ECDL - Makra
% Source from mail: ICT zavedena v rámci nějaké konkrétní vllnovité implementace


% TODO: Revise the last sentence. Maybe add some more suggar to the sentece.
Stejně jako bylo devatenácté století nazývané stoletím páry a dvacáté století stoletím techniky, bude jednou pravděpodobně dvacáté první století nazývané stoletím počítačů a sítí. Oblast techniky a informačních technologií se rozvíjí neustále vyšší a vyšší rychlostí, dostupnost \uv{chytré} elektroniky se zvyšuje tempem podobným. Společně s technikou a její vyšší dostupností se však objevuje další, bohužel nepříjemný, fenomén \-- a sice zneužívání technologií a jejich vysoké dostupnosti k páchání trestné činnosti. 

% Put footnote what it is
% Add source(s)
Trend zneužívání technologií a nedostatečné vzdělanosti v oblasti počítačové bezpečnosti lidí s nimi pracujících,  roste s každým rokem více a více. Útoky související s distribucí, v době psaní práce velmi populárního škodlivého software, zvaného ransomware vzrostly v počtu o neuvěřitelných 300$\%$.

 * Na jednu stranu možná i uklidňujícím faktem je, (cilem jsou firmy...pokračuj)

% TODO: Zjistit jak se píše "česká republika"
% TODO: Add source (http://www.nuv.cz/t/prehled-vydavani-rvp-sov-po-vlnach)
Reakcí českého školství na prudký rozvoj a expanzi výpočetní techniky byla samozřejmě snaha naučit žáky a studenty s těmito novými technologiemi pracovat a interagovat. Tato snaha byla realizována mezi lety 2007\--2009 v rámci vlnově vydávaných rámcově vzdělávacích programů \-- dále jen RVP \-- v nichž byl již povinně vyhrazen prostor pro výuku informačních a komunikačních technologií a informatiky. 





% Add specific examples of problematic events?
% 



---Technika se neustále se zrychlujícím rozvojem oblasti informatiky a zvyšující se dostupností elektroniky schopné připojovat se do Internetu, 


Tato sekce bude obsahovat argumenty pro zavedení výuky (základů) informační bezpečnosti na gymnázia, možná i (pro více textu) zmínku o tom, co tahle sekce obsahuje - klasické natahováky.
Bodově z konzultace: argumenty pro. 
Možné citace and témata: vzrůstající počet kybernetických útoků, důležitost vštípení žákům do hlavy jak nebezpečné je páchat trestnou činnost přes počítač, za co je můžou odsoudit, na jak dlouho a jak snadno pro blbosti.

\section{Kapitola 1 - RVP, ŠVP konkrétní školy, ...}
% TODO FILL THIS SECTION WITH SOMETHING USEFULL
Bez dalších poznámek, pravděpodobně půjde o rozbor toho co mi bude k dispozici (pokud nezískám ŠVP, asi se pustím do počtu hodin z RVP a zkusím vygooglit ŠVP nějaké školy, co to má veřejně). Patří sem informace o ŠVP/RVP, počet hodin na týden.

\section{Kapitola 2 - Vstupní dovednosti, ... to be clarified}
% TODO FILL THIS SECTION WITH SOMETHING USEFULL
Bodově z konzultace: předpokládané vstupní dovednosti, ve kterých ročnících by se to mělo vyučovat (předběžně domluveno na všechny), probrat možné formy výuky (výklad, exkurze, protip: přivedení člověka odsouzeného za trestnou činnost (část s vysvětlováním právní závadnosti)), formy výuky (co to vůbec je? :D), potřebné pomůcky pro realizaci: jak výkonné musí být počítače, jaké by měly být jejich parametry (asi internet je třeba), jde využít telefony (jak?), interaktivní tabule


% THIS IS PRACTICAL PART 
\section{Kapitola 3 - OBSAHOVÁ ČÁST - tématický plán výuky}
% TODO FILL THIS SECTION WITH SOMETHING USEFULL
Bez dalších poznámek, hádám, že tady si budu muset nastudovat jak vypadá tématický plán výuky a pokusit se přijít s vlastním (asi různý témata: spoofing mailů, nedůvěryhodnost spojení se stránkami (možnost napodobení jejich vzhledu), rozeznávání potenciálních útočníků (možná cool hra pro třídu - každému dát papírek s informací, kterou má z někoho zjistit, aniž by si toho ten někdo byl vědom; ne na hodinu ale jako domácí úkol na týden?), nevyžádaná pošta, právní stránka věci)


\section{Kapitola 4 - Osnova témat v jednotlivých ročnících}
% TODO FILL THIS SECTION WITH SOMETHING USEFULL
Bodově z konzultace: doplnit o didaktické hry (viz. výše), postupy a návody jak to učit. Jestli demonstrace, nebo jen teorie (vždycky demonstrace imho).

\section{Kapitola 5 - 4 ukázky příprav na hodinu}
% TODO FILL THIS SECTION WITH SOMETHING USEFULL
Asi potenciálně nejdelší část, hodně si s tím vyhraju, naplánuji časově co se v kolika hodinách bude dělat... ideální bude asi zakomponovat falšování mailu a ukázka jak snadné je udělat aby stránka vypadala stejně jako originální stránka, aniž by byla.

% SECTION CONCLUSION
\section{Závěr}
Zhodnocení celé práce, zhruba na 1 A4

\end{document}