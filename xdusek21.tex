% Set document properties as required by FEKT specification
\documentclass[a4paper, 12pt]{article}
\usepackage[left=3cm,right=2cm,top=4cm]{geometry}

% Local settings
\usepackage[utf8x]{inputenc} 
\usepackage[czech]{babel}
\usepackage[IL2]{fontenc}
\usepackage{afterpage}
\usepackage{amsthm}
\usepackage{amsmath}
\usepackage{amsfonts}
\usepackage{fancyvrb}
\usepackage{hyperref}
\hypersetup{
    colorlinks,
    citecolor=black,
    filecolor=black,
    linkcolor=black,
    urlcolor=black
}

% Set line spacing for whole document (sadly)
\renewcommand{\baselinestretch}{1.5}

% Helper commands
\providecommand{\uv}[1]{\quotedblbase #1\textquotedblleft}
\newcommand\textbox[1]{%
    \parbox{.5\textwidth}{#1}%
}

% Thesis variables
\newcommand{\thesisName}{Implementace kybernetické bezpečnosti do ŠVP pro Informatiku a výpočetní techniku na gymnáziích}
\newcommand{\universityName}{VYSOKÉ~~~UČENÍ~~~TECHNICKÉ~~~V~BRNĚ}
\newcommand{\facultyName}{Fakulta elektrotechniky a komunikačních technologií}
\title{\thesisname}
\author{Bc. Daniel Dušek}

% === WORK TODO LIST ===
% 
%
%
% PRAKTICKÁ ČÁST - TÉMATA, OSNOVY
% Topic: ECDL - Makra
% Source from mail: ICT zavedena v rámci nějaké konkrétní vllnovité implementace


\begin{document}

% === FRONT PAGE ===
\thispagestyle{empty}
\newgeometry{left=2cm,right=2cm,top=1.5cm}
    \begin{center}
        \Huge
        \universityName \\
            \vspace{\stretch{0.150}}
        
        \LARGE
        \textsc{\facultyName\\}
            \vspace{\stretch{0.300}}
        
        \Large{Závěrečná práce doplňujícího pedagogického studia \\ ~ \\}
        
        \LARGE
        \textsc{\thesisName}
            \vspace{\stretch{0.618}}
    \end{center}

        % Author and date part        
        \noindent \textbox{\today}  \textbox{\hfill \textbf{Vedoucí práce}: PhDr. Petra Fiľová} \\
        \noindent \textbox{\hfill}  \textbox{\hfill \textbf{Autor práce}: Bc. Daniel Dušek ~~~~~}
\clearpage
\restoregeometry

% === TABLE OF CONTENTS ===
% Potential TODO: NO line spacing as well?
\newpage
\thispagestyle{empty}
\tableofcontents

% === SECTION INTRODUCTION ===
\newpage
\setcounter{page}{1}
\section{Úvod}

Stejně jako bylo devatenácté století nazývané stoletím páry a dvacáté století stoletím techniky, bude jednou pravděpodobně dvacáté první století nazývané stoletím počítačů a sítí. Oblast techniky a informačních technologií se rozvíjí neustále vyšší a vyšší rychlostí, dostupnost \uv{chytré} elektroniky se zvyšuje tempem podobným. Společně s technikou a její vyšší dostupností se však objevuje další, bohužel nepříjemný fenomén -- a sice zneužívání technologií a jejich vysoké dostupnosti k páchání trestné činnosti. 

Trend zneužívání technologií a nedostatečné vzdělanosti v oblasti počítačové bezpečnosti lidí s nimi pracujících roste s každým rokem více a více. Útoky související s distribucí, v době psaní práce velmi populárního škodlivého software, zvaného ransomware vzrostly v počtu od roku 2015 neuvěřitelných 300$\%$.

Na jednu stranu možná i uklidňujícím faktem je, že v případě ransomware jsou cílem typicky organizace, nikoliv jednotlivci. Bohužel i v tomto případě se často obětí stane právě i jednotlivec, a to zejména díky způsobu, kterým se ransomware a jemu podobné škodlivé programy šíří. Jedním z častých scénářů je případ, kdy je počítač infikován prostřednictvím nakaženého souboru typu .xls(x), .doc(x), .ppt(x) apod., tedy soubory běžně produkované nástroji sady Microsoft Office. Druhým, opět velmi častým scénářem je pak případ, kdy se škodlivý soubor pouze tváří být souborem dříve jmenovaných typů, avšak ve skutečnosti je souborem úplně jiného typu, obsahujícího škodlivý kód. V obou dvou scénářích je klíčové, aby ovšem pochybil lidský článek pracující s takovýmto souborem. Tomuto pochybení by bylo poměrně snadné předcházet, a to vyšším obecným povědomím o tom jak správně a bezpečně pracovat s počítačem a internetem. 

Tohoto obecně vyššího povědomí by bylo možné dosáhnout například kladením vyššího důrazu na výuku kybernetické bezpečnosti ve školách v rámci výuky informačních technologií, výpočetní techniky a informatiky. V konkrétních termínech například místo kladení obrovského důrazu pouze na výuku práce s nástroji kancelářského balíku Microsoft Office slevit lehce z časové dotace vyhrazené na vysvětlování funkcionality a následné zkoušení žáků z ní, a tuto časovou úsporu věnovat k osvětlování způsobů a principů, kterými jsou soubory balíku Microsoft Office zneužívány hackery pro napadení počítačů a způsobů obrany a bezpečné práce s těmito soubory. Snížení časové dotace na vysvětlování a zkoušení funkcionality by, dle autora této práce, nemuselo mít vyloženě negativní dopad, neboť většina dnešní generace přichází do styku s počítači a těmito programy na denní bázi. Navíc jsou tyto programy v posledních letech průběžně neustále vylepšovány z hlediska uživatelské přívětivosti, aby práce s nimi byla více intuitivní a snadná. 

Dalším důvodem pro zavedení výuky kybernetické bezpečnosti do osnov je zcela opačný pól kybernetické bezpečnosti. Doposud bylo psáno hlavně o způsobech ochrany a osvěty běžných uživatelů výpočetní techniky proti útokům ze strany hackerů. Stejně, ne-li více důležitou částí výuky kybernetické bezpečnosti by měla být osvěta žáků o možných následcích zneužívání výpočetní techniky k páchání trestných činů. Mezi studenty se vyskytuje dnes již poměrně veliká část žáků, která je seznámena s prací s počítačem natolik, že je potenciálně schopná počítač i nějakým způsobem programovat. Někteří z těchto žáků mohou potřebovat vyjasnit fakt, že existuje hranice, za níž nesmí při programování svých aplikací zajít. Ačkoliv se to může zdát z prvního pohledu jako celkem jednoduchá otázka k diskusi, nemusí to vždy být pravda. Konkrétním případem může být 16 letý chlapec Ashkan Hosseini, jenž naprogramoval aplikaci, kterou nahrál na CD s rodinnými fotkami~\cite{malwareUnicornAppretince}. Aplikace následně způsobila, že po strčení CD do počítače došlo k vymazání všech rodinných fotek jak z CD, tak z počítače, do kterého bylo CD zastrčeno. Ashkan naprogramoval tuto aplikaci za účelem odstranění fotek na kterých se nacházel -- jeho úmysly tedy nebyly vyloženě špatné a díky tomu, že CD putovalo pouze mezi rodinou, Ashkan nebyl nikým žalován. Pokud by se ale stalo, že by jeho aplikace unikla do světa a napadla cizí počítač, dopustil by se již vážného trestného činu, aniž by si tyto následky svých akcí uvědomil. V Americe by to pro něho pak mohlo znamenat až 5 let vězení. Příběh Ashkana Hosseiniho dopadl tedy naštěstí pro něho dobře, kde navíc jeho rodiče měli dostatek reflexe a nabídli mu možnost studovat oblast boje proti kybernetické kriminalitě, které se chopil. Ne všichni žáci však musí vždy přesně odlišit hranici toho, kdy končí \uv{legrace} a začíná páchání trestného činu, alespoň ne v oblasti počítačů a sítí, a ne všichni musí mít to štěstí, jaké Ashkan měl. I z tohoto důvodu je třeba šířit povědomí o kybernetické bezpečnosti a implikacích, které může mít na život jedince nedodržování jejích zásad. 

V letech 2007--2009 byla reakcí českého školství na prudký rozvoj a expanzi výpočetní techniky snaha naučit žáky a studenty s těmito novými technologiemi pracovat a interagovat. Tato snaha byla realizována v rámci vlnově vydávaných rámcově vzdělávacích programů -- dále jen RVP -- v nichž byl již povinně vyhrazen prostor pro výuku informačních a komunikačních technologií a informatiky~\cite{waveRVP}. Autor této práce považuje tuto reakci za dobře načasovanou a vhodnou. 

Nyní však nastává nová éra, kdy zařízení schopná připojení k Internetu jsou doslova i obrazně na každém našem kroku. Každé zařízení připojené k Internetu se může stát cílem útoku, stejně jako se obětí počítačové kriminality může stát každá osoba s těmito zařízeními pracující. Vzniká tedy takto nová potřeba většího důrazu na výuku těchto aspektů práce s informačními technologiemi. Tato práce má za cíl demonstrovat konkrétní možnou implementaci takové výuky do výuky ICT na gymnáziích.

% === DICTIONARY ===
% TODO: Zkontrolovat, že všechna problematická slova se vyskytují ve slovníčku pojmů.
\newpage
\section{Slovníček pojmů}
V textu této práce je možné se setkat s pojmy, které mají speciální význam v oblasti počítačové či kybernetické bezpečnosti, jenž se často liší od pocitového významu, který by si oblasti neznalý čtenář mohl vytvořit. Tato sekce si klade za cíl rozptýlit možné nejasnosti a mnohoznačnosti.

\textbf{Malware} -- Souhrné označení pro programy které vykonávají škodlivou činnost (trojské koně, viry, programy špehující uživatele, programy zobrazující uživateli nevyžádanou reklamu, a další).

\textbf{Škodlivý kód} -- Chce-li programátor přikázat počítači, aby vykonal nějakou činnost, popíše tuto činnost pomocí tzv. \textit{kódu}. Škodlivý kód je pak takový kód, který je-li vykonán počítačem, provede něco, co uživatel/majitel tohoto počítače nechce. Útočníci se při útoku snaží typicky právě vykonat škodlivý kód na zařízení, které napadají. Tento škodlivý kód bývá velmi často nastražen a ukryt tak, aby si uživatel pracující s počítačem vůbec neuvědomil, že kód spouští a vykonává.

\textbf{Nakažený soubor} -- Soubor v němž je kromě původního obsahu souboru navíc obsažen škodlivý kód. Škodlivý kód do souboru umístil buď přímo, nebo zprostředkovaně (například prostřednictvím škodlivého programu) útočník.

\textbf{Ransomware} -- Škodlivý program založený na principu držení rukojmí. Program po průniku do počítače zašifruje některé soubory, popřípadě uživateli úplně znemožní práci s počítačem. Za zpřístupnění souborů či zařízení je po uživateli požadováno zaplacení výkupného. Zaplacení výkupného nemusí vždy vést k získání objektu, který je držen jako rukojmí.


% === RVP & SVP CHAPTER ===
% Doslova z papíru: Kap. RVP, ŠVP konkrétní školy - Informace - počet hodin / týden, Skoro kopie RVP/ŠVP s rozborem
\newpage
\section{Výuka informatiky dle Rámcového Vzdělávacího Programu a Školního Vzdělávacího Programu}
Tato část práce se zabývá prostorem vyhrazeným pro výuku informatiky v rámci Rámcového Vzdělávacího Programu (dále jen RVP) a jeho možnou utilizací pro výuku kybernetické bezpečnosti. Dále je zde rozebrán konkrétní Školní Vzdělávací Program (dále jen ŠVP) v rámci něhož je vyhrazen prostor pro výuku informatiky. Jsou zde také zmíněny očekávané výstupní dovednosti, kterými by student po absolvování vyučování informatiky měl disponovat.
%Bez dalších poznámek, pravděpodobně půjde o rozbor toho co mi bude k dispozici (pokud nezískám ŠVP, asi se pustím do počtu hodin z RVP a zkusím vygooglit ŠVP nějaké školy, co to má veřejně). Patří sem informace o ŠVP/RVP, počet hodin na týden.

% === PREREQUISITIES ===
\newpage
% TODO: Ocitovat RVP pro gymnázia & RVP pro základní školy
% TODO: Přijít na to jak správně pojmenovat kapitolu "Vstupní dovednosti"
% Očekávané výstupy – 1. a 2. období (základní vzdělávání)
% žák
% ICT-5-1-01 využívá základní standardní funkce počítače a jeho nejběžnější periferie
% ICT-5-1-02 respektuje pravidla bezpečné práce s hardwarem i softwarem a postupuje
%   poučeně v případě jejich závady
% ICT-5-1-03 chrání data před poškozením, ztrátou a zneužitím
\section{Vstupní dovednosti}
V této části práce jsou vymezeny vstupní dovednosti, kterými by studenti pro úspěšnou výuku kybernetické bezpečnosti měli disponovat. Dále je zde navrženo a argumentováno časové zasazení výuky napříč ročníky, pokryty jsou také možné formy výuky vhodné pro předmět kybernetické bezpečnosti a nezbytné pomůcky pro kvalitní výuku.
% Osnova
% * Vstupní dovednosti
% * Pomůcky pro výuku - u počítačů specifikovat operační systém, HW, to samé pro telefony, náhradní řešení bez telefonů
% * Časový rozvrh napříč ročníky
% * Formy výuky 
%           Bodově z konzultace: předpokládané vstupní dovednosti, ve kterých ročnících by se to mělo vyučovat (předběžně domluveno na všechny), probrat možné formy výuky (výklad, exkurze, protip: přivedení člověka odsouzeného za trestnou 
%           činnost (část s vysvětlováním právní závadnosti)), potřebné pomůcky pro realizaci: jak výkonné musí být počítače, jaké by měly být jejich parametry (asi internet je třeba), jde využít telefony (jak?), interaktivní tabule

\subsection{Vstupní dovednosti}
Text vstupních dovedností.

\subsection{Pomůcky pro výuku}
Test pomůcek ve výuce.

\subsection{Časový rozvrh napříč ročníky}
Text časového rozvrhu napříč ročníky

\subsection{Formy výuky}
Způsobů a forem, kterými lze vyučovat a demonstrovat kybernetickou bezpečnost by se jistě dalo vymyslet mnoho. Zde jsou shrnuty formy výuky, které autor práce považuje za efektivní a vhodné v oblasti kybernetické bezpečnosti, společně s argumentací využití právě těchto forem.

\subsubsection{Forma výuky: Výklad s živou ukázkou}
Text formy výuky.

\subsubsection{Forma výuky: Praktický domácí úkol, studenti kooperují}
Text formy výuky.

\subsubsection{Forma výuky: Návštěva člověka z praxe}
Text formy výuky.



% ========= PRACTICAL PART =========

\newpage
\section{Tématický plán výuky}
Na následujících stránkách jsou popsány dva možné tématické plány výuky kybernetické bezpečnosti na gymnáziích. První tématický plán uvažuje možnost výuky kybernetické bezpečnosti v rámci výuky hodin informačních technologí po její celou dobu stanovenou školou. Druhý tématický plán je realizován s předpokladem, že například v rámci 3. a 4. ročníku, je pro výuku kybernetické bezpečnosti dedikován samostatný předmět.

V rámci prnvího tématického plánu má výuka kybernetické bezpečnosti sloužit jako \textit{rozšíření} nebo \textit{doplnění}, které je možné vložit do standardní výuky informatiky realizované na gymnáziích. 

% TODO: Doplnit zdroj, že za 5 let nebude dost infosec specialistů.
V případě druhého plánu, narozdíl od výuky kybernetické bezpečnosti v rámci výuky informatiky umožňuje studentům se přímou soustředit a profilovat v oblasti kybernetické bezpečnosti. Další výhodou takovéto výuky je možnost podání látky v nahuštěnější formě, s vyšším důrazem na konkrétní problémy a jejich správné uchopení. V případě dedikovaného předmětu se není nutné vázat na probírané oblasti ve výuce standardní informatiky, ale je možné se věnovat konkrétním problémům kybernetické bezpečnosti a jejich řešení. Profilování se studentů v oblasti kybernetické bezpečnosti pro ně v dnešní době může být obzvláště zajímavé, neb se očekává, že rokem 2023 bude nedostatek kvalifikovaných specialistů v oblasti kybernetické (a informační) bezpečnosti.

% Výzkum: Tématický plán by měl obsahovat:
% * název předmětu, 
% * hodinovou dotaci (třeba zjistit jak správně tuto dotaci poznačovat),
% * časové rozvržení výuky,
% * pravidla hodnocení,
% * učebnici,
% * pomůcky (redundance? Nebo tady s tím vymyslím něco jako nutný seznam pomůcek; nebo na základě toho, že to má být v tématickém plánu si vynutím sekci pomůcky znovu)
% Učivo pro hodiny má být sestaveno na základě ŠVP
% Bez dalších poznámek, hádám, že tady si budu muset nastudovat jak vypadá tématický plán výuky a pokusit se přijít s vlastním (asi různý témata: spoofing mailů, nedůvěryhodnost spojení se stránkami (možnost napodobení jejich vzhledu), rozeznávání potenciálních útočníků (možná cool hra pro třídu - každému dát papírek s informací, kterou má z někoho zjistit, aniž by si toho ten někdo byl vědom; ne na hodinu ale jako domácí úkol na týden?), nevyžádaná pošta, právní stránka věci)


\newpage
\section{Osnova témat v jednotlivých ročnících}
Tato část práce popisuje rozložení témat napříč jednotlivými ročníky a jejich náročnost. Jsou zde opět uvedeny dvě varianty -- pro výuku kybernetické bezpečnosti v rámci výuky informatiky a pro výuku kybernetické bezpečnosti jako samostatného předmětu.
% Opět dvě verze: Pro dedikovaný předmět || Pro předmět připlácnutý k informatice
%Bodově z konzultace: doplnit o didaktické hry (viz. výše), postupy a návody jak to učit. Jestli demonstrace, nebo jen teorie (vždycky demonstrace imho).

\newpage
%Asi potenciálně nejdelší část, hodně si s tím vyhraju, naplánuji časově co se v kolika hodinách bude dělat... ideální bude asi zakomponovat falšování mailu a ukázka jak snadné je udělat aby stránka vypadala stejně jako originální stránka, aniž by byla.
\section{Ukázky příprav na hodiny}
V této kapitole práce jsou prezentovány celkem čtyři ukázky možných příprav na hodiny výuky kybernetické bezpečnosti.

\subsection{Příprava 1 -- Krádeže identity}
% TODO: Pokud všude bude využívána stejná struktura, přesunout tento výčet výše
%Text přípravy na hodinu s podvrháváním emailu. Přihlašování se do různých místností pod různými přihlašovacími jmény. Pravděpodobně i stránky s podobnými názvy a stejným designem, ale jiným přijemcem dat sem spadají.
Příprava na hodinu s tématem \textbf{Krádeže identity} je rozdělena do následujících částí:
    \begin{itemize}
        \setlength{\itemsep}{-3pt}
        \item Specifikace tématu hodiny,
        \item stanovené cíle,
        %\item plán ověřování vědomostí a dovedností,
        \item náplň hodiny a časový plán,
        \item nutné pomůcky pro hodinu,
        %\item zadání nepovinného domácího úkolu.
    \end{itemize}

Struktura rozložení přípravy na hodinu vychází z doporučené obecné struktury v dokumentu \textit{Příprava učitele na výuku. Vzdělávání učitelů.}~\cite{presentationPavlaZ}, kde tato obecná struktura je upravena tak, aby odpovídala přípravě na hodinu kybernetické bezpečnosti.

\subsubsection{Specifikace tématu hodiny}
Hodina se zabývá tématem krádeže identity na Internetu, riziky situací, kdy se člověk stane obětí krádeže identity (aktivní či pasivní oběť) a nejčastějšími typy identit, které jsou odcizovány, nebo jsou podvrhnutelné. V hodině by také měl být popsán a demonstrován způsob, kterým ke zcizení identity může dojít, včetně uvedení možných právních následků takové činnosti.

\subsubsection{Stanové cíle}
Studenti by po hodině měli rozumět pojmu \textit{krádež identity} a měli by chápat rizika spojená s touto hrozbou. Budou znát kanály, kterými ke krádeži identity může dojít a budou chápat, že jde o na Internetu o velmi běžnou záležitost. Dále by měli být znát právní následky, kterým mohou čelit v případě že se dopustí krádeže identity, a co vše je možné považovat za krádež identity. V hodině dojde k demonstraci jak snadné je podvrhnout email z téměř jakékoliv emailové schránky, žáci by měli být schopni podvržení emailu zopakovat. Zároveň budou žáci schopní u jakéhokoliv emailu, který dorazí do jejich emailové schránky rozhodnout na základě probraného algoritmu, zda je bezpečné email otevřít, či zda je lepší ho smazat. Existují-li ve třídě žáci, kteří se věnují informatice, nebo mají větší zájem o problematiku efektivní obrany proti podvržení emailové identity, budou jim poskytnuty doplňující zdroje informací, řešící problém.

\subsubsection{Náplň hodiny a časový plán}
\indent\textbf{0.--3. minuta} - Zápis do třídní knihy, v případě, že výuka probíhá v počítačové učebně, zapnutí počítačů a přihlášení žáků do systému.

\textbf{4.--10. minuta} - Představení tématu krádeže identity, představení očekávané náplně hodiny, podáno formou slovní monologické metody.

\textbf{11.--30. minuta} - Frontální formou výuky kombinovanou s monologickým výkladem, vysvětlení žákům co to je krádež identity, jak se člověk může aktivně i pasivně stát obětí této techniky. Vyučující může pokládat otázky přímo žákům ohledně kanálů, po kterých může dojít k zneužití této techniky a pomáhat jim přijít s reálnými případy. Dále vyučující zdůrazní riziko podvržení cizí identity při emailové komunikaci, názorně demonstruje jak snadné to je -- pokud se hodina odehrává v počítačové učebně, vybere z řad studentů dobrovolníka, který se přihlásí na učitelském počítači do emailu a vyučující z telefonu odešle podvržený email do schránky žáka, kde se, se svolením žáka, bude vydávat právě za žáka sedícího za počítačem. Následně třídě vysvětlí, jak velmi důležité v celém demonstračním scénáři bylo, aby dostal k odeslání emailu pod identitou žáka u počítače svolení. Plynule tak může přejít k vysvětlení právní závadnosti podvrhávání emailů a vysvětlí žákům, jakým následkům by mohli čelit, pokud by se pokusili podvržení emailové identity využít k nečestným účelům. Odehrává-li se výuka ve třídě, kde je převaha dívek, je možné uvést příklad ze seriálu \textit{Prolhané krásky (anglicky Pretty Little Liars)}, kde jedna z hlavních postav, Caleb, právě podvržení emailové identity zneužije, aby mu byla zkrácena doba, co musí strávit po škole. V rámci tohoto časového okna také vyučující žákům vysvětlí jednoduchý algoritmus, podle kterého se mohou rozhodovat, zda přílohu, popřípadě celý email, je možné bezpečně otevřít a nebo je lepší ho smazat. Algoritmus je založen na myšlence odpovědi na otázku: \uv{Očekávám v tento čas, od tohoto odesilatele tuto zprávu?}, je-li odpověď ne, pak je lepší přílohu emailu nikdy neotevírat a spojit se s odesilatelem alternativním způsobem, například zavolat a ověřit, že email skutečně posílali odesilatel.

\textbf{31.--40. minuta} - Pokud žáci mezi 11.-30. minutou nedošli po navádějících otázkách učitele na to, že vytvoření stránky, která vypadá úplně stejně jako jiná stránka za účelem získávání dat uživatelů stránky, jejíž vzhled napodobují, vyučující tuto možnost krádeže identity zmíní. Pokud na ni žáci došli v předchozím bloku sami, zopakuje to ještě jednou, mimochodem se žákům zmíní, že této technice se velmi často říká \textit{phishing} a vysvětlí jim, jak se bránit tomu, aby se nestali obětí -- s využitím jednoduché principu zkontrolování adresy webové stránky, na které se nachází pokaždé, než na ní vyplní nějaké důvěrné údaje (heslo, uživatelské jméno, email).

\textbf{41.--45. minuta} - Existují-li žáci s větším zájmem o informatiku a specificky možnou obranu proti krádeži emailové identity, může je vyučující nasměrovat k samostudiu technologií pod zkratkami \textit{DMARC}, \textit{SPF}, \textit{DKIM}. Dále vyučující, bude-li ve třídě zájem, zadá volitelný domácí úkol, kde jeho předmětem bude doručit do schránky vyučujícího umístěné ve službě \textit{gmail.com} podvržený email s přesně zadanými náležitostmi (emailová identita držená taktéž vyučujícím), který nebude doručen do složky SPAM. První ze studentů, který uspěje může získat známku za aktivitu/bonusové body do předmětu.

\subsubsection{Nutné pomůcky pro hodinu}
Pro splnění stanových cílů hodiny je nutné, aby v místnosti, ve které má být hodina vyučována, byl přítomen počítač s připojením k internetu, s data projektorem. Vyučující by si s sebou měl přinést telefon, nebo notebook, taktéž s přístupem k Internetu. Alespoň 2 zařízení s přístupem k Internetu, kdy jedno ze zařízení je schopné posílat obraz na data projektor jsou nutné k výše popsaným demonstracím.



\subsection{Příprava 2 -- }
Text přípravy na hodinu.

\subsection{Příprava 3 -- }
Text přípravy na hodinu.

\subsection{Příprava 4 -- }
Text přípravy na hodinu.

% SECTION CONCLUSION
\newpage
\section{Závěr}
%Zhodnocení celé práce, zhruba na 1 A4 - still missing







% === REFERENCES & LITERATURE ===
\newpage
\begin{thebibliography}{9}

    \bibitem{malwareUnicornAppretince}
    Selena Larsen: Malware researcher helps teen hackers turn skills into careers,
    \\\textit{http://money.cnn.com/2017/07/12/technology/malware-researcher-helps-teen-hackers}.

    \bibitem{waveRVP}
    Národní Ústav pro Vzdělávání: Přehled vydávání RVP SOV po vlnách,
    \\\textit{http://www.nuv.cz/t/prehled-vydavani-rvp-sov-po-vlnach}.

    \bibitem{presentationPavlaZ}
    ZIELENIECOVÁ, Pavla. \textit{Příprava učitele na výuku. Vzdělávání učitelů.} [online prezentace]. Praha: Katedra didaktiky fyziky, Matematicko-fyzikální fakulta, UK, 2015, [cit. 2017-19-08]. Dostupný z WWW: $<$https://kdf.mff.cuni.cz/vyuka/pedagogika/materialy/2015\%20ZS/8\%20Priprava\\\%20ucitele\%20na\%20vyuku.\%20Legislativni\%20zakotveni\%20ucitele.pdf$>$.

\end{thebibliography}
\end{document}